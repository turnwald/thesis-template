\chapter{Background and Related Work}
\label{chap:background}

This chapter summarizes the theoretical background and related work relevant to the problem addressed in this thesis. The purpose is not to provide an exhaustive survey, but to establish the necessary foundations and terminology.

\section{Fundamental Concepts}

Many engineering problems can be described using abstract system models, algorithms, or data representations. Depending on the application domain, this may include physical modeling, optimization methods, or learning-based approaches.

Figure~\ref{fig:background_overview} illustrates a generic conceptual structure that is commonly encountered in the literature.

\begin{figure}[ht]
	\centering
	\includegraphics[width=0.8\textwidth]{fig/loop.png}
	\caption{Generic conceptual overview of the considered problem domain.}
	\label{fig:background_overview}
\end{figure}

\section{Related Work}

Previous work has addressed similar problems from different perspectives. Table~\ref{tab:related_work} provides a high-level comparison of representative approaches discussed in the literature.

\begin{table}[h]
	\centering
	\caption{Comparison of representative approaches from related work.}
	\label{tab:related_work}
	\begin{tabular}{lccc}
		\toprule
		Approach & Model-based & Data-driven & Experimental validation \\
		\midrule
		Method A & yes & no  & yes \\
		Method B & no  & yes & yes \\
		Method C & yes & yes & no  \\
		\bottomrule
	\end{tabular}
\end{table}

The comparison highlights that existing approaches typically involve trade-offs between interpretability, performance, and implementation complexity.
